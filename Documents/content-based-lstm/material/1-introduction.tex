%%%%%%%%%%%%%%%%%%%%%%%%%%%%%%%%%%%%%%%%
%%%%% INTRODUCTION
%%%%%%%%%%%%%%%%%%%%%%%%%%%%%%%%%%%%%%%%
\section{INTRODUCTION}
In recent years, the influence of the music recommendation system on the recording industry has become much more significant than it was before due to the emergence of online music streaming services. The current recommendation systems utilize a collaborative filtering approach on the metadata of music to characterize songs. While this method is a very intuitive way to estimate users' preference, it raises several problems for users and music artists if the recommendation system solely relies on metadata. Because songs are characterized by metadata such as artists, the songs by popular artists are much more likely to be recommended to users. This is problematic for mainly two reasons. 

First, it limits the user's ability to discover new artists and songs. The purpose of browsing songs is generally to discover more likable songs beyond artists or genres. If the recommender system keeps suggesting songs that have already been liked or artists the user follows, it may conflict with the user's interests. Furthermore, users may never get an opportunity to cultivate the interests towards other types of artists or songs in different genres if only similar songs are being suggested. Second, it creates an unfair disadvantage for unknown artists. The top 1\% of artists account for the vast majority of all revenue from recording music. \cite{top-artist-site} While it is true that the top artists are more prolific than majority of artists, the music recommendation system that solely based on the content of music can raise the visibility of undiscovered, talented music artists.

Several recent studies addresses such a problem, but more research can be done on the subject. Hence, the goal of our research is to provide an additional metric that can improve the quality of music recommendation systems. In order to eliminate the problem discussed above, this paper proposes a method of analyzing music and users' preference that is solely based on raw music data. Furthermore, we propose an alternative approach to understand users' preference of songs. Traditional music recommender generally only predicts whether or not users would like the song. However, such a metric cannot tell how much users would enjoy the song. In this paper, we utilized Long Short-Term Memory regression model to estimate the music play counts by users. The music play counts are often proportional to how much user enjoys a song, and it is useful to predict songs that users are likely to enjoy the most. In addition, such a recommender feature could be potential application for industrial use, since the interests of distribution companies and music artists is how many times listeners would play their songs.
