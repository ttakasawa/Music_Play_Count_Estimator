%%%%%%%%%%%%%%%%%%%%%%%%%%%%%%%%%%%%%%%%
%%%%% CONCLUSION
%%%%%%%%%%%%%%%%%%%%%%%%%%%%%%%%%%%%%%%%
\section{CONCLUSION}
Most research in the field of music recommendation relies on binary classification or categorizing music into clusters. While there has been success in this area, many of these methods still rely heavily on metadata to accomplish their goal.\cite{Wang2014,Wang:2014:ICH:2647868.2654940} Although the success of our experiments was relatively minor considering the constraints that time and budget imposed on the amount of data that was able to be processed, the results were fairly promising and indicate that further research can be fruitful.

As stated, our goal was to find a method to assist current music recommendation systems while being metadata agnostic. Agnosticism of information such as the musician(s) responsible for writing a song creates potential for new artists to be recommended equally as well established ones, in essence creating a level playing field in the market and a lower barrier to entry. The content-based nature of the neural network creates an environment in which latent features may be extracted regardless of genre or artist and theoretically allow users to expand the repertoire of music that they enjoy. In addition, the focus on predicting the number of times a user will play a song potentially allows the distribution service, record label, and music artists to increase revenue.