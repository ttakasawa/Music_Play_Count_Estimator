\documentclass[paper=a4, fontsize=11pt]{scrartcl} % A4 paper and 11pt font size

%----------------------------------------------------------------------------------------
%	PACKAGES
%----------------------------------------------------------------------------------------
\usepackage[T1]{fontenc} % Use 8-bit encoding that has 256 glyphs
%\usepackage{fourier} % Use the Adobe Utopia font for the document - comment this line to return to the LaTeX default
\usepackage[english]{babel} % English language/hyphenation
\usepackage{amsmath,amsfonts,amsthm} % Math packages
\usepackage{sectsty} % Allows customizing section commands
\usepackage{fancyhdr} % Custom headers and footers
\usepackage{tabularx, outlines, framed, varwidth, enumitem, graphicx, listings, color, qtree, float, subcaption, newfloat, natbib}

\usepackage[left=0.5in, right=0.5in, top=3in, bottom=.25in]{geometry}
\geometry{}

%----------------------------------------------------------------------------------------
%	SET CUSTOMIZATIONS AND FUNCTIONS
%----------------------------------------------------------------------------------------
\sectionfont{\centering \normalfont\scshape} % Make all sections centered, the default font and small caps
\pagestyle{fancyplain} % Makes all pages in the document conform to the custom headers and footers
\fancyhead{} % No page header - if you want one, create it in the same way as the footers below
\fancyfoot[L]{} % Empty left footer
\fancyfoot[C]{} % Empty center footer
\fancyfoot[R]{\thepage} % Page numbering for right footer
\renewcommand{\headrulewidth}{0pt} % Remove header underlines
\renewcommand{\footrulewidth}{0pt} % Remove footer underlines
\setlength{\headheight}{0pt} % Customize the height of the header

\DeclareFloatingEnvironment[fileext=lod]{diagram}

\numberwithin{equation}{section} % Number equations within sections (i.e. 1.1, 1.2, 2.1, 2.2 instead of 1, 2, 3, 4)
\numberwithin{figure}{section} % Number figures within sections (i.e. 1.1, 1.2, 2.1, 2.2 instead of 1, 2, 3, 4)
\numberwithin{table}{section} % Number tables within sections (i.e. 1.1, 1.2, 2.1, 2.2 instead of 1, 2, 3, 4)

\graphicspath{{./}}
%\setlength\parindent{0pt} % Removes all indentation from paragraphs - comment this line for an assignment with lots of text

\makeatletter
	\newcommand*\variableheghtrulefill[1][.4\p@]
	{%
		\leavevmode
		\leaders \hrule \@height #1\relax \hfill
		\null
	}
\makeatother

\lstset
{
	language=C++,
%	basicstyle=\ttfamily,
%	keywordstyle=\color{blue}\ttfamily,
%	stringstyle=\color{red}\ttfamily,
%	commentstyle=\color{green}\ttfamily,
%	morecomment=[l][\color{magenta}]{\#}
	keywordstyle=\color{blue},
	stringstyle=\color{red},
	commentstyle=\color{green},
	morecomment=[l][\color{magenta}]{\#}
}

%----------------------------------------------------------------------------------------
%	USEFUL COMMANDS
%----------------------------------------------------------------------------------------
%	\makebox[\textwidth][c]{\includegraphics[width=.9\pagewidth]{p2-table}}

%	\newgeometry{top=.75in, bottom=.75in, left=.25in,right=.25in}
%	\newgeometry{top=.75in, bottom=.75in, left=1.25in,right=1.25in}

%	\lstinputlisting[firstline=4]{CMPSC360_Homework.cpp}

%\Tree
%	[.<root> [.<left> ][.<middle> ][.<right> ]]

%----------------------------------------------------------------------------------------
%	TITLE SECTION
%----------------------------------------------------------------------------------------

\newcommand{\horrule}[1]{\rule{\linewidth}{#1}} % Create horizontal rule command with 1 argument of height
% \title{Template: Homework 1}
\title{	
\normalfont \normalsize 
%\textsc{Rutgers University, Real Analysis I} \\ [25pt] % Your university, school and/or department name(s)
\horrule{0.5pt} \\[0.4cm] % Thin top horizontal rule
\huge DS 340: Music Recommendation Using Unsupervised Deep Learning \\ % The assignment title
\horrule{2pt} \\[0.5cm] % Thick bottom horizontal rule
}

\author{Kyle Salitrik | Tomoki Takasawa} % Your name

\date{\normalsize\today} % Today's date or a custom date

\begin{document}

\maketitle % Print the title

%----------------------------------------------------------------------------------------
%	Body
%----------------------------------------------------------------------------------------
\renewcommand{\outlineii}{enumerate} % Set level 2 of outline to numbers
% http://tug.ctan.org/macros/latex2e/contrib/outlines/outlines.pdf

\newgeometry{top=.75in, bottom=.75in, left=1.25in,right=1.25in}
\section*{\variableheghtrulefill[.25ex]\quad Project Description \& Purpose \quad\variableheghtrulefill[.25ex]}
For our project we intend to create a way to find similarities between songs in a user's music library without using provided labels. This unsupervised approach will hopefully be able to provide the likelihood that a user will enjoy a song based solely on the music itself. The purpose of being genre, artist, and popularity agnostic in decision making for the recommendations is to introduce the users to new music that they may not normally enjoy. 

The intended application would be to scan a user's music library as training data and then scan a wide library of songs to find ones with features similar to the user's current library with a tunable width of variability. These songs will be recommended to the user to listen to. Ideally the user may be able to respond whether or not they actually enjoyed the song in order to adjust future recommendations.

\section*{\variableheghtrulefill[.25ex]\quad Who Will Benefit \quad\variableheghtrulefill[.25ex]}
Artists, users, and music providers would be able to benefit from successful results of our application. Not taking into account the popularity of a song or artist would allow new or unknown artists to be discovered. As genres are not considered -- only the music itself -- users will be able to benefit by expanding their music tastes. 

\section*{\variableheghtrulefill[.25ex]\quad Frameworks to be Used \quad\variableheghtrulefill[.25ex]}
We plan to use following open-source libraries and data sets will be used to accomplish our task
\begin{itemize}
	\setlength{\itemsep}{1pt}
	\setlength{\parskip}{0pt}
	\setlength{\parsep}{0pt}
	\item LibROSA
	\item Keras
	\item TensorFlow
    \item LabROSA One-Million Song Data Set
    
\end{itemize}

\section*{\variableheghtrulefill[.25ex]\quad Milestones \quad\variableheghtrulefill[.25ex]}
\subsection*{Week 6}
We plan to accomplish followings by the end of Week 6

\begin{outline}
	\1 Collecting data
    \1 Designing the architecture of the project
    \1 Extracting some of the features
    	\2 frequency
    		\3 frequency band
            \3 frequency power
        \2 repetition
        	\3 rhythm
            \3 lyrics
    
\end{outline}
\subsection*{Week 9}
\begin{outline}
    \1 Extracting all the features
    	\2 tempo
        \2 musical keys
    \1 Potentially applying existing Cocktail Party Effect networks
    	\2 masking noises (other instruments)
\end{outline}

\section*{\variableheghtrulefill[.25ex]\quad Initial Resources \quad\variableheghtrulefill[.25ex]}
	\nocite{*}
	\bibliographystyle{plainnat}
	\bibliography{DS340}
\end{document}